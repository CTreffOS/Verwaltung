\documentclass[a4paper,12pt]{scrartcl}
\usepackage[utf8]{inputenc}
\usepackage[T1]{fontenc}
\usepackage[ngerman]{babel}
\usepackage{libertine} % kann man notfalls auch ignorieren, wenns nicht da ist
\usepackage{textcomp}  % für Euro-Symbol

\title{Beitragsordnung des Chaostreff Osnabrück e.V.}
\author{ctreffos.de}
\date{23.~März~2015}

\begin{document}
\maketitle

\section*{\S{} 1 Beitragssätze}
\begin{enumerate}
\item[(1)]
Der reguläre Mitgliedsbeitrag für ordentliche Mitglieder beträgt 60 €
pro Jahr. Fördermitglieder zahlen einen frei wählbaren Beitrag von
mindestens 60 € pro Jahr.
\item[(2)]
Schüler, Studenten, Auszubildende, Empfänger von Sozialgeld oder
Arbeitslosengeld~II einschließlich Leistungen nach §~22 ohne Zuschläge
oder nach §~24 des Zweiten Buchs des Sozialgesetzbuchs (SGB~II) sowie
Empfänger von Ausbildungsförderung nach dem
Bundesausbildungsförderungsgesetz (BAföG) haben die Möglichkeit, für
die ordentliche Mitgliedschaft einen ermäßigten Beitrag von 30 € pro
Jahr zu zahlen. Ein entsprechender Nachweis muss dem Vorstand auf
Verlangen zugänglich gemacht werden.
\item[(3)]
Sollte ein ordentliches Mitglied aus finanziellen Gründen den
Mitgliedsbeitrag nicht aufbringen können, kann dieses beim Vorstand
einen Antrag auf Ermäßigung oder Befreiung stellen. Diese gilt
zunächst für maximal ein Jahr und kann durch Folgeanträge für jeweils
maximal ein weiteres Jahr verlängert werden.
\item[(4)]
Alle Mitglieder werden ermutigt, im Rahmen ihrer Möglichkeiten eine
regelmäßige Spende an den Verein zu entrichten. Empfohlen wird eine
Spende in Höhe von 1\% des Bruttoeinkommens.
\end{enumerate}

\section*{\S{} 2 Fälligkeit}
\begin{enumerate}
\item[(1)]
Der Mitgliedsbeitrag ist jährlich im Voraus fällig.
\item[(2)]
Bei Beitritt eines Mitgliedes während des laufenden Geschäftsjahres ist der Beitrag 
anteilig zu entrichten, beginnend mit dem Monat, in den der Tag der Aufnahme fällt.
\end{enumerate}

\section*{\S{} 3 Zahlungsweise}
\begin{enumerate}
\item[(1)]
Die Zahlung des Mitgliedsbeitrages sollte per Überweisung (z.B.\
Dauerauftrag) erfolgen.
\item[(2)]
Alternativ zu Abs. 1 kann eine Barzahlung an 
den Schatzmeister, zu einem von ihm festgelegten Termin, erfolgen.
\end{enumerate}

\section*{\S{} 4 Aufnahmegebühren}
Es wird eine Aufnahmegebühr in Höhe von 0 € erhoben.

\end{document}
