\documentclass[a4paper,12pt]{scrartcl}
\usepackage[utf8]{inputenc}
\usepackage[T1]{fontenc}
\usepackage[ngerman]{babel}
\usepackage{libertine} % kann man notfalls auch ignorieren, wenns nicht da ist
\usepackage{textcomp}  % für Euro-Symbol

\title{Beitragsordnung des Chaostreff Osnabrück~e.V.}
\date{23.~März~2015}

\begin{document}
\maketitle

\section{Beitragssätze}
\begin{enumerate}
  \item Der reguläre Mitgliedsbeitrag für ordentliche Mitglieder beträgt 60€
    pro Jahr. Fördermitglieder zahlen einen frei wählbaren Beitrag von
    mindestens 120€ pro Jahr.
  \item Schüler, Studenten, Auszubildende, Empfänger von Sozialgeld oder
    Arbeitslosengeld~II einschließlich Leistungen nach §~22 ohne Zuschläge oder
    nach §~24 des Zweiten Buchs des Sozialgesetzbuchs (SGB~II), sowie Empfänger
    von Ausbildungsförderung nach dem Bundesausbildungsförderungsgesetz (BAföG)
    haben die Möglichkeit, für die ordentliche Mitgliedschaft einen ermäßigten
    Beitrag von 30€ pro Jahr zu zahlen. Ein entsprechender Nachweis muss dem
    Vorstand auf Verlangen zugänglich gemacht werden.
  \item Sollte ein ordentliches Mitglied aus finanziellen Gründen den
    Mitgliedsbeitrag nicht aufbringen können, kann dieses beim Vorstand einen
    Antrag auf Ermäßigung oder Befreiung stellen. Diese gilt für maximal ein
    Jahr und kann dann durch einen neuen Antrag erneuert werden.
  \item Alle Mitglieder werden ermutigt, im Rahmen ihrer Möglichkeiten eine
    regelmäßige Spende an den Verein zu entrichten. Empfohlen wird eine Spende
    in Höhe von 1\% des Bruttoeinkommens.
\end{enumerate}

\section{Fälligkeit}
\begin{enumerate}
  \item Der Mitgliedsbeitrag wird jeweils zum Ende des laufenden Mitgliedsjahres bzw. mit der Annahme des Aufnahmeantrags fällig.
\end{enumerate}

\section{Zahlungsweise}
\begin{enumerate}
  \item\label{item:ueberweisung} Die Zahlung des Mitgliedsbeitrages kann per
    Überweisung (z.~B. Dauerauftrag) erfolgen.
  \item Alternativ zu Abs.~\ref{item:ueberweisung} kann auch eine Barzahlung an
    den Schatzmeister erfolgen, sofern dieser zum entsprechenden Zeitpunkt dazu
    bereit ist.
\end{enumerate}

\section{Aufnahmegebühren}
\begin{enumerate}
  \item Es wird eine Aufnahmegebühr von 10€ erhoben.
  \item Die Aufnahmegebühr ist bei Annahme der Aufnahmeantrags zu entrichten.
\end{enumerate}

\end{document}