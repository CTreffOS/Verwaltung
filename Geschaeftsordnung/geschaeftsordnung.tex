\documentclass[a4paper,12pt]{scrartcl}
\usepackage[utf8]{inputenc}
\usepackage[T1]{fontenc}
\usepackage[ngerman]{babel}
\usepackage{libertine} % kann man notfalls auch ignorieren, wenns nicht da ist
\usepackage{textcomp}  % für Euro-Symbol

\title{Geschäftsordnung des Vorstands des Chaostreff Osnabrück e.V.}
\author{ctreffos.de}
\date{23.~März~2015}

\begin{document}
\maketitle

\section*{\S{} 1 Interne Aufgaben- und Zuständigkeitsverteilung}
\begin{enumerate}
\item[(1)]
Der Vorstand beschließt intern folgende Aufgaben und Zuständigkeitsverteilung:

\begin{enumerate}
\item \textbf{Vorstand:}
\begin{enumerate}
\item Einberufen der Vorstandssitzungen
\item Leiten der Vorstandssitzungen
\item Lädt zur Mitgliederversammlung
\end{enumerate}

\item \textbf{Stellvertreter:}
\begin{enumerate}
\item Führung des Protokolls
\item Koordination der Projekte
\item Erhalt der Gemeinnützigkeit
\end{enumerate}

\item \textbf{Schatzmeister:}
\begin{enumerate}
\item Verwaltet des Vermögen des Vereins
\item Überwacht die Zahlungs Ein- und Ausgänge
\item Führt die Mitgliederliste
\item Bedient die Verbindlichkeiten des Vereins
\item Organisiert das Mahnwesen
\item Rechenschaftslegung
\end{enumerate}

\item \textbf{Chaosbeauftragter:}
\begin{enumerate}
\item Ist für Öffentlichkeitsarbeit zuständig
\item Austausch mit Chaos-Familie \& befreundeten Organisationen
\item Vertretung auf Regio-Treffs des CCC e.V.
\end{enumerate}

\end{enumerate}

\item[(2)]
Falls ein Vorstandsmitglied den Internen Aufgaben (vgl. oben) aufgrund von Abwesenheit, Krankheit etc. nicht wahrnehmen kann, gilt folgenden Regelung:
\begin{enumerate}
\item Der Vorstandsvorsitzende wird durch den 1. Stellvertreter vertreten.
\item Der 1. Stellvertreter wird durch den 2. Stellvertreter vertreten.
\item Der Schatzmeister wird durch den Chaosbeauftragten vertreten.
\item Der Chaosbeauftragten wird durch den Schatzmeister vertreten.
\end{enumerate}

\end{enumerate}

\section*{\S{} 3 Vorstandssitzungen}
\begin{enumerate}
\item[(1)]
Die Vorstandssitzungen finden bei Bedarf, mindestens jedoch einmal pro Quartal, statt.
\item[(2)]
Die Sitzungen werden durch den Vorsitzenden unter Angabe der Tagesordnung in
Textform einberufen.
\item[(3)]
In dringenden Fällen oder wenn mindestens zwei Vorstandsmitglieder dies gegenüber
des Vorsitzenden verlangen, finden außerordentliche Vorstandssitzungen statt.
\item[(4)]
Der Vorstand ist Beschlussfähig wenn mindestens 3 Vorstandsmitglieder anwesend
sind.
\item[(5)]
Der Vorstand trifft seine Beschlüsse mit der absoluten Mehrheit der Anwesenden
\item[(6)]
Alle Abstimmungen finden per Handzeichen statt.
\item[(7)]
Die Tagesordnung wird vom Vorsitzenden erstellt. Vorschläge der Vorstandsmitglieder
sind von ihm zu berücksichtigen. Sie enthält damit alle Anträge, die dem Vorsitzenden
vorgelegt werden. Die Tagesordnungspunkte können bei Bedarf verändert werden.


\end{enumerate}

\end{document}
