\documentclass[12pt,paper=a4,ngerman]{scrreprt}
\usepackage{babel}
\usepackage[T1]{fontenc}
\usepackage[utf8]{inputenc}
\usepackage{hyperref}
\usepackage{eurosym}
\setcounter{secnumdepth}{0}
\setcounter{tocdepth}{3}
\hypersetup{colorlinks, linkcolor=black}
\title{Satzung\\des\\Chaostreff Osnabrück} \author{ctreffos.de}
\date{23. März 2015}

\begin{document}
\maketitle
\tableofcontents
\newpage

\section{Präambel}

Die Informationsgesellschaft unserer Tage ist ohne Computer nicht mehr
denkbar. Die Einsatzmöglichkeiten der automatisierten
Datenverarbeitung und Datenübermittlung bergen Chancen, aber auch
Gefahren für den Einzelnen und für die Gesellschaft. Infor\-mations-
und Kommunikationstechnologien verändern das Verhältnis
Mensch-Maschine und der Menschen untereinander. \\
Die Entwicklung zur Informationsgesellschaft erfordert ein neues
Menschenrecht auf weltweite, ungehinderte Kommunikation. Der
Chaostreff Osnabrück ist eine Gemeinschaft, unabhängig von Alter,
Geschlecht und Abstammung sowie gesellschaftlicher Stellung, die sich
grenzüberschreitend für Informationsfreiheit einsetzt und mit den
Auswirkungen von Technologien auf die Gesellschaft sowie das einzelne
Lebewesen beschäftigt und das Wissen um diese Entwicklung fördert.

\section{\S{} 1 Name, Sitz, Geschäftsjahr}

\begin{itemize}
\item[(1)]
Der Verein führt den Namen \glqq Chaostreff
Osnabrück\grqq. Der Verein wird in das Vereinsregister eingetragen und
dann um den Zusatz \glqq e.V.\grqq ~ergänzt. Der Verein hat seinen
Sitz in Osnabrück.
\item[(2)]
Das Geschäftsjahr ist das Kalenderjahr.
\end{itemize}

\section{\S{} 2 Zweck und Gemeinnützigkeit}
\begin{itemize}
\item[(1)]
Der Verein fördert und unterstützt Vorhaben der Bildung und
Volksbildung in Hinsicht neuer technischer Entwicklungen sowie Kunst
und Kultur im Sinne der Präambel oder führt diese durch. Der
Vereinszweck soll unter anderem durch folgende Mittel erreicht
werden:
\begin{enumerate}
\item
Regelmäßige öffentliche Treffen und Informationsveranstaltungen.
\item
Veranstaltungen und/oder Förderung internationaler Kongresse, Treffen,
Konferenzen sowie virtueller Zusammenkünfte.
\item
Herausgabe von Publikationen in digitaler oder gedruckter Form.
\item
Öffentlichkeitsarbeit und Telepublishing in allen Medien.
\item
Arbeits- und Erfahrungsaustauschkreise.
\item
Informationsaustausch mit den in der Datenschutzgesetzgebung
vorgesehenen Kontrollorganen.
\item
Förderung des schöpferisch-kritischen Umgangs mit Technologie.
\item
Hilfestellung und Beratung bei technischen und rechtlichen Fragen im
Rahmen der gesetzlichen Möglichkeiten für die Mitglieder.
\item
Zusammenarbeit und Austausch mit nationalen und internationalen
Gruppierungen, deren Ziele mit denen des Vereins vereinbar sind.
\item
Veranstaltungen und Projekte, die sich speziell an Jugendliche richten.
\end{enumerate}
\item[(2)]
Der Verein verfolgt ausschließlich und unmittelbar gemeinnützige
Zwecke im Sinne des Abschnitts \glqq Steuerbegünstigte Zwecke\grqq\
der Abgabenordnung 1977 (§ 51 ff. AO) in der jeweils gültigen
Fassung; er dient ausschließlich und unmittelbar der Volksbildung zum
Nutzen der Allgemeinheit. Er darf keine Gewinne erzielen, er ist
selbstlos tätig und verfolgt nicht in erster Linie
eigenwirtschaftliche Zwecke. Die Mittel des Vereins werden
ausschließlich und unmittelbar zu den satzungsgemäßen Zwecken
verwendet. Die Mitglieder erhalten keine Zuwendung aus den Mitteln des
Vereins. Niemand darf durch Ausgaben, die dem Zwecke des Vereins fremd
sind oder durch unverhältnismäßig hohe Vergütungen begünstigt werden.
\end{itemize}

\section{\S{} 3 Mitgliedschaft}
\begin{itemize}
\item[(1)]
Ordentliche Vereinsmitglieder können ausschließlich natürliche
Personen werden.
\item[(2)]
Die Beitrittserklärung erfolgt in Textform gegenüber dem Vorstand.
Über die Annahme der Beitrittserklärung entscheidet der Vorstand. Die
Mitgliedschaft beginnt mit der Annahme der Beitrittserklärung und der
Zahlung der Aufnahmegebühr.
\item[(3)]
Die Mitgliedschaft endet durch Austrittserklärung, durch den Tod von
natürlichen Personen oder durch Auflösung und Erlöschung von
juristischen Personen, Handelsgesellschaften, nicht rechtsfähigen
Vereinen sowie Anstalten und Körperschaften des öffentlichen Rechts
oder durch Ausschluss; die Beitragspflicht für das laufende
Geschäftsjahr bleibt hiervon unberührt.
\item[(4)]
Der Austritt wird durch Willenserklärung in Textform gegenüber dem
Vorstand vollzogen.
\item[(5)]
Die Mitgliederversammlung kann solche natürliche Personen, die sich
besondere Verdienste um den Verein oder um die von ihm verfolgten
satzungsgemäßen Zwecke erworben haben, zu Ehrenmitgliedern ernennen.
Ehrenmitglieder haben alle Rechte eines ordentlichen Mitglieds; sie
sind von Beitragsleistungen befreit.
\item[(6)]
Fördermitglieder sind passive Mitglieder ohne Stimmrecht in der
Mitgliederversammlung. Fördermitglieder können natürliche und
juristische Personen, nicht rechtsfähige Vereine sowie Anstalten und
Körperschaften des öffentlichen Rechts werden. Bei Minderjährigen ist
die Zustimmung des gesetzlichen Vertreters erforderlich.
\end{itemize}

\section{\S{} 4 Rechte und Pflichten der Mitglieder}
\begin{itemize}
\item[(1)]
Die Mitglieder sind berechtigt, die Leistungen des Vereins in Anspruch
zu nehmen.
\item[(2)]
Die Mitglieder sind verpflichtet, die satzungsgemäßen Zwecke des
Vereins zu unterstützen und zu fördern. Sie sind verpflichtet, die
festgesetzten Beiträge zu zahlen.
\end{itemize}

\section{\S{} 5 Ausschluss eines Mitglieds}
\begin{itemize}
\item[(1)]
Ein Mitglied kann durch Beschluss des Vorstandes ausgeschlossen
werden, wenn es das Ansehen des Vereins schädigt, seinen
Beitragsverpflichtungen nicht nachkommt oder wenn ein sonstiger
wichtiger Grund vorliegt. Der Vorstand muss dem auszuschließenden
Mitglied den Beschluss in Textform unter Angabe von Gründen mitteilen
und ihm auf Verlangen eine Anhörung gewähren.
\item[(2)]
Gegen den Beschluss des Vorstandes ist die Anrufung der
Mitgliederversammlung zulässig. Bis zum Beschluss der
Mitgliederversammlung ruht die Mitgliedschaft.
\end{itemize}

\section{\S{} 6 Beitrag}
\begin{itemize}
\item[(1)]
Der Verein erhebt einen Aufnahme- und Jahresbeitrag. Er ist bei der
Aufnahme bzw. im Voraus zu entrichten. Das Nähere regelt eine
Beitragsordnung, die von der Mitgliederversammlung beschlossen wird.
Im Falle nicht fristgerechter Entrichtung der Beiträge ruht die
Mitgliedschaft.
\item[(2)]
Im begründeten Einzelfall kann für ein Mitglied durch
Vorstandsbeschluss ein von der Beitragsordnung abweichender Beitrag
festgesetzt werden.
\end{itemize}

\section{\S{} 7 Organe des Vereins}
Die Organe des Vereins sind:
\begin{enumerate}
\item
die Mitgliederversammlung,
\item
der Vorstand.
\end{enumerate}

\section{\S{} 8 Mitgliederversammlung}
\begin{itemize}
\item[(1)]
Oberstes Beschlussorgan ist die Mitgliederversammlung. Ihrer
Beschlussfassung unterliegen:
\begin{enumerate}
\item die Genehmigung des Finanzberichtes,
\item die Entlastung des Vorstandes,
\item die Wahl der einzelnen Vorstandsmitglieder,
\item die Bestellung von Finanzprüfern,
\item die Satzungsänderungen,
\item die Genehmigung der Beitragsordnung,
\item die Richtlinie über die Erstattung von Reisekosten und Auslagen,
\item die Anträge des Vorstandes und der Mitglieder,
\item die Ernennung von Ehrenmitgliedern,
\item die Auflösung des Vereins.
\end{enumerate}
\item[(2)]
Die ordentliche Mitgliederversammlung findet jährlich statt.
Außerordentliche Mitgliederversammlungen werden auf Beschluss des
Vorstandes abgehalten, wenn die Interessen des Vereins dies erfordern,
oder wenn mindestens zehn Mitglieder dies unter Angabe des Zwecks
schriftlich beantragen. Die Einberufung der Mitgliederversammlung
erfolgt in Textform durch den Vorstand mit einer Frist von mindestens
zwei Wochen. Zur Wahrung der Frist reicht die Aufgabe der Einladung
zur Post an die letzte bekannte Anschrift oder die Versendung an die
zuletzt bekannte E-Mail-Adresse.\\
Hierbei sind die Tagesordnung bekanntzugeben und die nötigen
Informationen zugänglich zu machen. Anträge zur Tagesordnung sind
mindestens sieben Tage vor der Mitgliederversammlung bei dem Vorstand
in Textform einzureichen. Über die Behandlung von Initiativanträgen
entscheidet die Mitgliederversammlung.
\item[(3)]
Die Mitgliederversammlung ist beschlussfähig, wenn mindestens fünf
Prozent aller Mitglieder, die nicht dem Vorstand angehören, anwesend
sind. Beschlüsse sind jedoch gültig, wenn die Beschlussfähigkeit vor
der Beschlussfassung nicht angezweifelt worden ist. Ist die
Mitgliederversammlung aufgrund mangelnder Teilnehmerzahl nicht
beschlussfähig, ist die darauf folgende ordentlich einberufene
Mitgliederversammlung ungeachtet der Teilnehmerzahl beschlussfähig.
\item[(4)]
Beschlüsse über Satzungsänderungen und über die Auflösung des Vereins
bedürfen zu ihrer Rechtswirksamkeit der Dreiviertelmehrheit der
anwesenden stimmberechtigten Mitglieder. In allen anderen Fällen
genügt die einfache Mehrheit.
\item[(5)]
Jedes Mitglied, welches mit den Beiträgen nicht im Rückstand ist, hat
eine Stimme. Stimmen können nicht übertragen werden.
\item[(6)]
Auf Antrag eines Mitglieds ist geheim abzustimmen. Über die Beschlüsse
der Mitgliederversammlung ist ein Protokoll anzufertigen, das vom
Versammlungsleiter und dem Protokollführer zu unterzeichnen ist. Das
Protokoll ist allen Mitgliedern zugänglich zu machen und auf der
nächsten Mitgliederversammlung genehmigen zu lassen.
\item[(7)]
Die Mitgliederversammlung wählt den Vorstand und die Finanzprüfer. Die
Wahlen finden geheim in Form der \glqq Wahl durch Zustimmung\grqq\
statt. Jeder Wähler kann beliebig vielen Kandidaten jeweils eine
Stimme geben. Jeder zu besetzende Posten wird einzeln gewählt, wobei
gleichrangige Posten (die zwei stellvertretenden Vorsitzenden und die
zwei Finanzprüfer) jeweils gemeinsam gewählt werden. Bei der Wahl des
Vorsitzenden, des Schatzmeisters und des Chaos-Beauftragen ist
gewählt, wer die meisten abgegebenen Stimmen erhält. Bei
Stimmengleichheit findet eine Stichwahl statt. Bei erneuter
Stimmengleichheit entscheidet das Los. Bei der Wahl der
stellvertretenden Vorsitzenden und der Finanzprüfer sind diejenigen
beiden Kandidaten gewählt, die die meisten Stimmen erhalten. Bei
Stimmengleichheit findet eine Stichwahl statt. Bei erneuter
Stimmengleichheit entscheidet das Los.
\end{itemize}

\section{\S{} 9 Vorstand}
\begin{itemize}
\item[(1)]
Der Vorstand besteht aus fünf Mitgliedern:
\begin{enumerate}
\item
dem Vorsitzenden,
\item
zwei stellvertretenden Vorsitzenden,
\item
dem Schatzmeister und
\item
dem Chaos-Beauftragen.
\end{enumerate}
\item[(2)]
Jedes Vorstandsmitglied ist berechtigt, den Verein nach außen zu
vertreten. Ausgenommen sind Einstellung und Entlassung von
Angestellten, gerichtliche Vertretung sowie Anzeigen, Aufnahme von
Krediten, Gründung, Erwerb und Veräußerung von Gesellschaften und
Geschäftsanteilen von Gesellschaften zur Verwirklichung der
satzungsgemäßen Ziele; bei denen der Verein durch mindestens drei
Vorstandsmitglieder vertreten wird.
\item[(3)]
Sind zwei oder mehr Vorstandsmitglieder dauernd an der Ausübung ihres
Amtes gehindert, so sind unverzüglich Nachwahlen anzuberaumen.
\item[(4)]
Die Amtsdauer der Vorstandsmitglieder beträgt zwei Jahre. Wiederwahl
ist zulässig. Damit auch nach Ablauf der Amtsdauer eine ordnungsgemäße
gesetzliche Vertretung gesichert ist, bleibt der Vorstand bis zur
Neuwahl im Amt.
\item[(5)]
Der Vorstand ist Dienstvorgesetzter aller vom Verein angestellten
Mitarbeiter; er kann diese Aufgabe einem Vorstandsmitglied
übertragen.
\item[(6)]
Der Schatzmeister überwacht die Haushaltsführung und verwaltet das
Vermögen des Vereins. Er hat auf eine sparsame und wirtschaftliche
Haushaltsführung hinzuwirken. Mit dem Ablauf des Geschäftsjahres
stellt er unverzüglich die Abrechnung sowie die Vermögensübersicht und
sonstige Unterlagen von wirtschaftlichem Belang den Finanzprüfern des
Vereins zur Prüfung zur Verfügung.
\item[(7)]
Die Vorstandsmitglieder sind grundsätzlich ehrenamtlich tätig; sie
haben Anspruch auf Erstattung notwendiger Auslagen im Rahmen einer von
der Mitgliederversammlung zu beschließenden Richtlinie über die
Erstattung von Reisekosten und Auslagen.
\item[(8)]
Der Vorstand kann \glqq Fachliche Beiräte\grqq\ oder \glqq
Wissenschaftliche Beiräte\grqq\ einrichten, die für den Verein
beratend und unterstützend tätig werden; in die Beiräte können auch
Nichtmitglieder berufen werden.
\item[(9)]
Der Vorstand gibt sich eine Geschäftsordnung, die von der
Mitgliederversammlung zu genehmigen ist.
\end{itemize}

\section{\S{} 10 Finanzprüfer}
\begin{itemize}
\item[(1)]
Zur Kontrolle der Haushaltsführung bestellt die Mitgliederversammlung
zwei Finanzprüfer. Nach Durchführung ihrer Prüfung informieren sie den
Vorstand von ihrem Prüfungsergebnis und erstatten der
Mitgliederversammlung Bericht.
\item[(2)]
Die Finanzprüfer dürfen dem Vorstand nicht angehören.
\item[(3)]
Die Finanzprüfer sind grundsätzlich ehrenamtlich tätig; sie haben
Anspruch auf Erstattung notwendiger Auslagen im Rahmen einer von der
Mitgliederversammlung zu beschließenden Richtlinie über die Erstattung
von Reisekosten und Auslagen.
\end{itemize}

\section{\S{} 11 Auflösung des Vereins}
Bei der Auflösung des Vereins oder bei Wegfall seines Zweckes fällt
das Vereinsvermögen an eine von der Mitgliederversammlung zu
bestimmende Körperschaft des öffentlichen Rechts oder eine andere
aufgrund ihrer Gemeinnützigkeit steuerbegünstigten Körperschaft zwecks
Verwendung für die Volksbildung.

\end{document}
